\chapter{Аналитический раздел}

\section{Постановка задачи}

В соответствии с техническим заданием на курсовую работу, необходимо разработать веб-приложения для обмена данными через локальный почтовый сервис.

Для решения поставленной задачи необходимо:
\begin{enumerate}
	\item Провести анализ протокола SMTP.
	\item Разработать алгоритмы, реализующие протокол SMTP.
	\item Разработать библиотеку, реализующую SMTP-сервер.
	\item Разработать библиотеку, реализующую SMTP-клиент.
	\item Разработать веб-приложение для обмена данными через почтовый сервис.
\end{enumerate}

\section{Требования к разрабатываемому ПО}

В соответствии с техническим заданием на курсовую работу, были установлены следующие требования к разрабатываемому программному обеспечению:
\begin{enumerate}
	\item SMTP-сервер должен быть запущен локально.
	\item Для получения сообщения сервер должен использовать разработанную библиотеку.
	\item Для отправки сообщения веб-приложение должно использовать разработанную библиотеку.
	\item Веб-приложение должно, помимо отправки сообщения, предусматривать авторизацию и регистрацию пользователей, просмотр входящих и исходящих сообщений, удаление сообщений.
\end{enumerate}

\section{Анализ протокола SMTP}

SMTP (\textit{англ.} \textbf{S}imple \textbf{M}ail \textbf{T}ransfer \textbf{P}rotocol — простой протокол передачи почты) — требующий соединения текстовый протокол, по которому отправитель сообщения связывается с получателем посредством выдачи командных строк и получения необходимых данных через надёжный канал, в роли которого обычно выступает TCP-соединение (по умолчанию — 25 порт).
SMTP-сессия состоит из команд, посылаемых SMTP-клиентом, и соответствующих ответов SMTP-сервера.
Когда сессия открыта, сервер и клиент обмениваются её параметрами. Сессия может включать ноль и более SMTP-операций (транзакций).

SMTP-операция состоит из трёх последовательностей команда/ответ (см. пример ниже).
Описание последовательностей:
\begin{itemize}
	\item \code{MAIL FROM} — устанавливает обратный адрес (то есть Return-Path, 5321.From, mfrom).
	Это адрес для возвращённых писем.
	\item \code{RCPT TO} — устанавливает получателя данного сообщения.
	Эта команда может быть дана несколько раз, по одной на каждого получателя.
	Эти адреса также являются частью оболочки.
	\item \code{DATA} — для отправки текста сообщения.
	Это само содержимое письма, в противоположность его оболочке.
	Он состоит из заголовка сообщения и тела сообщения, разделённых пустой строкой.
	\code{DATA}, по сути, является группой команд, а сервер отвечает дважды: первый раз на саму команду \code{DATA}, для уведомления о готовности принять текст; и второй раз после конца последовательности данных, чтобы принять или отклонить всё письмо.
\end{itemize}

Помимо промежуточных ответов для DATA-команды, каждый ответ сервера может быть положительным (код ответа 2хх) или отрицательным. Последний, в свою очередь, может быть постоянным (код 5хх) либо временным (код 4хх).
Отказ SMTP-сервера в передаче сообщения — постоянная ошибка; в этом случае клиент должен отправить возвращённое письмо.
После сброса — положительного ответа, сообщение скорее всего будет отвержено.
Также сервер может сообщить о том, что ожидаются дополнительные данные от клиента (код 3xx).

Согласно \href{https://tools.ietf.org/html/rfc5321#section-4.5.1}{секции 4.5.1 RFC 5321} помимо трёх перечисленных, любой SMTP сервер должен поддерживать команды:
\begin{itemize}
	\item \code{EHLO}, \code{HELO} — приветствие, начало сессии.
	Предпочтительнее использовать \code{EHLO}.
	\item \code{RSET}.
	Эта команда указывает, что текущая почтовая транзакция будет прервана.
	\item \code{NOOP}.
	Эта команда не влияет на параметры или ранее введенные команды. Он не определяет никаких действий, кроме отправки получателем ответа \code{250 OK}.
	\item \code{QUIT}.
	Эта команда указывает, что получатель ДОЛЖЕН отправить ответ \code{221 OK}, а затем закрыть канал передачи.
	\item \code{VRFY}.
	Эта команда просит получателя подтвердить, что аргумент идентифицирует пользователя или почтовый ящик.
\end{itemize}

\section{Выводы}

В результате анализа технического задания на курсовую работу была поставлена задача, были определены основные требования к разрабатываемому ПО.

В результате анализа протокола SMTP были определены принципы его работы, что позволяет перейти к проектированию библиотек, реализующих SMTP-сервер и SMTP-клиент.