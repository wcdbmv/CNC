\chapter{Аналитический раздел}

\section{Постановка задачи}

В соответствии с техническим заданием на курсовую работу, необходимо разработать веб-приложения для обмена данными через локальный почтовый сервис.

Для решения поставленной задачи необходимо:
\begin{enumerate}
	\item Провести анализ протокола SMTP.
	\item Разработать алгоритмы, реализующие протокол SMTP.
	\item Разработать библиотеку, реализующую SMTP-сервер.
	\item Разработать библиотеку, реализующую SMTP-клиент.
	\item Разработать веб-приложение для обмена данными через почтовый сервис.
\end{enumerate}

\section{Требования к разрабатываемому ПО}

В соответствии с техническим заданием на курсовую работу, были установлены следующие требования к разрабатываемому программному обеспечению:
\begin{enumerate}
	\item SMTP-сервер должен быть запущен локально.
	\item Для получения сообщения сервер должен использовать разработанную библиотеку.
	\item Для отправки сообщения веб-приложение должно использовать разработанную библиотеку.
	\item Веб-приложение должно, помимо отправки сообщения, предусматривать авторизацию и регистрацию пользователей, просмотр входящих и исходящих сообщений, удаление сообщений.
\end{enumerate}

\section{Анализ протокола SMTP}

SMTP (англ. Simple Mail Transfer Protocol --- простой протокол передачи почты) --- это сетевой протокол, предназначенный для передачи электронной почты между сервером отправителя и почтовым клиентом или сервером получателя через надёжный канал, в роли которого обычно выступает TCP-соединение.
Взаимодействие в рамках SMTP строится по принципу двусторонней связи, которая устанавливается между отправителем и получателем почтового сообщения. При этом отправитель инициирует соединение и посылает запросы на обслуживание, а получатель отвечает на эти запросы.

Операция протокола включает в себя комбинацию, состоящую из следующих последовательностей команд и ответов:

\begin{itemize}
	\item MAIL FROM --- команда, устанавливающая обратный электронный адрес;
	\item RCPT TO --- команда, определяющая получателя письма;
	\item DATA --- команда, отвечающая за отправку текста электронного сообщения --- это тело письма, которое включает в себя заголовок и текста письма, разделенные пустой строкой.
\end{itemize}

Помимо промежуточных ответов для DATA-команды, каждый ответ сервера может быть положительным или отрицательным. Последний, в свою очередь, может быть постоянным либо временным. Отказ SMTP-сервера в передаче сообщения --- постоянная ошибка. В этом случае клиент должен отправить возвращённое письмо.
После сброса, то есть положительного ответа, сообщение скорее всего будет отвержено. Кроме того сервер может сообщить о том, что ожидаются дополнительные данные от клиента.

Согласно \href{https://tools.ietf.org/html/rfc5321#section-4.5.1}{секции 4.5.1 RFC 5321} помимо трёх перечисленных, любой SMTP сервер должен поддерживать команды:
\begin{itemize}
	\item \code{EHLO}, \code{HELO} — команда-приветствие, обозначает начало сессии.
	Предпочтительнее использование команды \code{EHLO}.
	\item \code{RSET} --- команда, указывающая, что текущая почтовая транзакция будет прервана.
	\item \code{NOOP} --- команда, не влияющая на параметры или ранее введенные команды. Она не определяет никаких действий, кроме отправки получателем ответа 250 OK.
	\item \code{QUIT} --- команда, указывающая, что получатель должен отправить ответ 221 OK, а затем закрыть канал передачи.
	\item \code{VRFY} --- команда, запрашивающая у получателя подтверждение, что аргумент идентифицирует пользователя или почтовый ящик.
\end{itemize}

\section{Обзор существующих решений}

Рассмотрим некоторые существующие решения, на базе которых можно организовать почтовую связь в локальной сети.

Sendmail --- агент передачи почты с открытым исходным кодом, разработанный Эриком Оллманом в 1983 году. Является кросс-платформенным, поддерживает множество способов аутентификации. К достоинством данного программного обеспечения можно отнести портативность и гибкость, а к недостаткам --- сложность модификации, слабые механизмы безопасности.

Postfix --- свободное программное обеспечение, которое разрабатывалось как альтернатива Sendmail. Данный агент передачи почты кросс-платформенный, ориентирован, в первую очередь, на безопасность, имеет исчерпывающую документацию, обеспечивает высокую скорость работы. Postfix полностью совместим с Sendmail. К недостаткам можно отнести сложность при настройке и поддержке сервиса.

Exim --- это агент пересылки почты, который обычно используется в Unix-подобных операционных системах. Имеет монолитную архитектуру. Достоинства сервиса: гибкость конфигурации, большое сообщество, совместимость с Sendmail. К недостатка можно отнести сложность работы в сравнении с другими агентами, монолитную архитектуру ПО.

\section{Выводы}

В результате анализа технического задания на курсовую работу была поставлена задача, были определены основные требования к разрабатываемому ПО.

В результате анализа протокола SMTP были определены принципы его работы, что позволяет перейти к проектированию библиотек, реализующих SMTP-сервер и SMTP-клиент.

Кроме того, были рассмотрены существующие решения, определены их достоинства и недостатки.