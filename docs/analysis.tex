\chapter{Аналитический раздел}

\section{Постановка задачи}

В соответствии с техническим заданием на курсовую работу, необходимо разработать веб-приложения для обмена данными через локальный почтовый сервис. 

Для решения поставленной задачи необходимо:
\begin{enumerate}
	\item Провести анализ протокола SMTP.
	\item Разработать алгоритмы, реализующие протокол SMTP.
	\item Разработать библиотеку, реализующую SMTP-сервер.
	\item Разработать библиотеку, реализующую SMTP-клиент.
	\item Разработать веб-приложение для обмена данными через почтовый сервис.
\end{enumerate}

\section{Требования к разрабатываемому ПО}

В соответствии с техническим заданием на курсовую работу, были установлены следующие требования к разрабатываемому программному обеспечению:
\begin{enumerate}
	\item SMTP-сервер должен быть запущен локально.
	\item Для получения сообщения сервер должен использовать разработанную библиотеку.
	\item Для отправки сообщения веб-приложение должно использовать разработанную библиотеку.
	\item Веб-приложение должно, помимо отправки сообщения, предусматривать авторизацию и регистрацию пользователей, просмотр входящих и исходящих сообщений, удаление сообщений.
\end{enumerate}

\section{Анализ протокола SMTP}

\section{Выводы}

В результате анализа технического задания на курсовую работу была поставлена задача, были определены основные требования к разрабатываемому ПО.

В результате анализа протокола SMTP были определены принципы его работы, что позволяет перейти к проектированию библиотек, реализующих SMTP-сервер и SMTP-клиент.