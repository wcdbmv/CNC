\chapter*{Введение}
\addcontentsline{toc}{chapter}{Введение}

Технология электронной почты появилась в 1965 году --- сотрудники Массачусетского технологического института Ноэль Моррис и Том Ван Влек написали программу mail для операционной системы CTSS. Однако коммерческое использование электронной почты началось только в 1990-х с запуском сервиса Hotmail.

Общепринятым в мире протоколом обмена электронной почтой является протокол SMTP, который использует DNS для определения правил пересылки почты. Стандарт протокола был впервые описан в 1982 году (RFC 821), а затем дополнен в 2008 году (RFC 5321). Почтовые серверы используют SMTP для отправки и получения почтовых сообщений. Работающие на уровне пользователя клиентские почтовые приложения обычно используют SMTP только для отправки писем на почтовый сервер, а для получения сообщений применяют другие протоколы (POP, IMAP).

Курсовая работа посвящена разработке веб-приложения для обмена данными через локальный почтовый сервис и библиотек, реализующих SMTP-сервер и SMTP-клиент.