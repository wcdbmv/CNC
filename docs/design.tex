\chapter{Конструкторский раздел}

\section{Проектирование библиотек}

\subsection{Библиотека, реализующая SMTP-сервер}

Библиотека должна представлять интерфейс для реализации SMTP-сервера.

Необходимо реализовать перечисленные в аналитической части команды протокола SMTP согласно спецификации RFC 5321.

SMTP-сервер можно описать конечным автоматом состояний.
Диаграмма состояний SMTP-сервера представлена на рисунке \ref{img:smtp-state-diagram}.

\imgwc{h}{\textwidth}{smtp-state-diagram}{Диаграмма состояний SMTP-сервера}

Синтаксис команд представлен в листинге \ref{lst:syntax}.

\begin{lstlisting}[gobble=8, caption={Синтаксис поддерживаемых команд\label{lst:syntax}}]
	EHLO <hostname>
	HELO <hostname>
	MAIL FROM: <address>
	RCPT TO: <address>
	DATA
	RSET
	NOOP
	QUIT
\end{lstlisting}

Помимо вышеописанных команд, библиотека должна предоставлять интерфейс функции обработки сообщения, которая позволит произвести журналирование, запись в базу данных и прочее.

\subsection{Библиотека, реализующая SMTP-клиент}

Для работы через протокол SMTP клиент создаёт TCP соединение с сервером.
Затем клиент и SMTP-сервер обмениваются информацией пока соединение не будет закрыто или прервано.

Библиотека должна поддерживать как интерфейс команд протокола SMTP отдельно, так и единую функцию, инкапсулирующую работу с сервером.
Такая функция на вход будет принимать адрес электронной почты отправителя, список адресов электронных почт получателей, заголовок письма и его содержимое.

Схема режима работы SMTP-клиента представлена на рисунке \ref{img:smtp-client-trace}.

\imgwc{h}{0.8\textwidth}{smtp-client-trace}{Схема режима работы SMTP-клиента}

\section{Проектирование веб-приложения}

\subsection{База данных}

Прежде всего необходимо спроектировать базу данных для хранения информации о сообщениях и пользователях почтового сервиса.

Каждый пользователь веб-приложения описывается именем, электронным адресом и паролем.

Каждое сообщение описывается электронным адресом отправителя и получателей, темой письма, временем отправки либо получения письма и содержанием письма.

Кроме того, необходимы дополнительные сущности, описывающие наличие письма во входящем или исходящем ящиках для обеспечения возможности удаления.

На рисунке \ref{img:db} представлена схема базы данных.

\imgwc{h}{0.6\textwidth}{db}{Схема базы данных}

\subsection{SMTP-сервер}

SMTP-сервер должен использовать разработанную библиотеку.

В перегруженной функции обработки сообщения необходимо произвести следующие действия:
\begin{enumerate}
	\item сохранить сообщение в таблицу Message;
	\item сохранить запись исходящей почты в таблице MessageFrom;
	\item для каждого получателя сохранить запись входящей почты в таблице MessageTo;
\end{enumerate}

\subsection{Архитектура приложения}

Для проектирования приложения применим архитектурый шаблон проектирования Model-View-Controller (далее — MVC).

\subsubsection{Архитектурный шаблон MVC}

MVC представляет из себя схему разделения данных и бизнес-логики приложения, пользовательского интерфейса и управляющей логики на три независимых компонента: модель, представление и контроллер. Такой подход позволяет изолировать данные и управляющую логику, независимо разрабатывать, тестировать, поддерживать и модифицировать компоненты. Схема шаблона MVC представлена на рисунке \ref{img:mvc}.

\imgwc{h}{0.6\textwidth}{mvc}{Диаграмма шаблона MVC}

Модель представляет собой данные и методы для работы с данными. В модели выполняются запросы к базе данных, бизнес-логика. Этот компонент разрабатывается таким образом, чтобы отвечать на запросы контроллера, изменять свое внутреннее состояние и не зависеть от представлений.

Представление получает данные модели и отображает их пользователю. Представление не обрабатывает данные.

Контроллер является связующим компонентом --- интерпретирует действия пользователя, оповещая модель об изменениях, которые необходимо внести.

\subsubsection{UML-диаграмма компонентов приложения}

На рисунке \ref{img:appd} представлена UML-диаграмма компонентов веб-приложения.

\imgwc{h}{0.7\textwidth}{appd}{UML-диаграмма компонентов приложения}

\section{Выводы}

В конструкторском разделе были спроектированы библиотеки, реализующие протокол SMTP. Были спроектированы база данных и архитектура приложения, построена UML-диаграмма компонентов веб-приложения.