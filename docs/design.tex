\chapter{Конструкторский раздел}

\section{Проектирование библиотек}

\subsection{Библиотека, реализующая SMTP-сервер}

\subsection{Библиотека, реализующая SMTP-клиент}

\section{Проектирование веб-приложения}

\subsection{База данных}

Для хранения сообщений необходимо разработать базу данных.

Каждый пользователь веб-приложения описывается именем, электронным адресом и паролем.

Каждое сообщение описывается электронным адресом отправителя, темой письма, временем отправки/получения письма и содержанием письма.

Кроме того, необходима дополнительная сущность, описываемая идентификатором письма и электронным адресом получателя, так как одно письмо может быть отправлено нескольким адресатам.

На рисунке \ref{img:db} представлена диаграмма базы данных.

\imgwc{h}{0.6\textwidth}{db}{Диаграмма базы данных}

\subsection{SMTP-сервер}

\subsection{Архитектура приложения}

Для проектирования приложения применим архитектурый шаблон проектирования Model-View-Controller (далее — MVC).

\subsubsection{Архитектурный шаблон MVC}

MVC представляет из себя схему разделения данных и бизнес-логики приложения, пользовательского интерфейса и управляющей логики на три независимых компонента: модель, представление и контроллер. Такой подход позволяет изолировать данные и управляющую логику, независимо разрабатывать, тестировать, поддерживать и модифицировать компоненты. Схема шаблона MVC представлена на рисунке \ref{img:mvc}.

\imgwc{h}{0.6\textwidth}{mvc}{Диаграмма шаблона MVC}

Модель представляет собой данные и методы для работы с данными. В модели выполняются запросы к базе данных, бизнес-логика. Этот компонент разрабатывается таким образом, чтобы отвечать на запросы контроллера, изменять свое внутреннее состояние и не зависеть от представлений.

Представление получает данные модели и отображает их пользователю. Представление не обрабатывает данные.

Контроллер является связующим компонентом --- интерпретирует действия пользователя, оповещая модель об изменениях, которые необходимо внести.

\subsubsection{UML-диаграмма компонентов приложения}

На рисунке \ref{img:appd} представлена UML-диаграмма компонентов веб-приложения.

\imgwc{h}{0.8\textwidth}{appd}{UML-диаграмма компонентов приложения}

\section{Выводы}

В конструкторском разделе были спроектированы библиотеки, реализующие протокол SMTP. Были спроектированы база данных и архитектура приложения, построена UML-диаграмма компонентов веб-приложения.