\chapter{Технологический раздел}

\section{Выбор языка программирования и среды разработки}

В качестве языка программирования был выбран язык Python. 

В качестве фреймворка для разработки веб-приложения был выбран фреймворк Django. Django является одним из наиболее популярных фреймворков для разработки веб-приложений на Python. В качестве СУБД был выбран SQLite, который используется в Django по умолчанию.

В качестве среды разработки была выбрана IDE PyCharm. PyCharm содержит редактор кода, отладчик, средства для статического анализа кода, средства для сборки проекта, тесную интеграцию с фреймворком Django.

\section{Реализация библиотек}

\subsection{Библиотека, реализующая SMTP-сервер}

\subsection{Библиотека, реализующая SMTP-клиент}

\section{Реализация веб-приложения}

\subsection{Реализация SMTP-сервера}

\subsection{Реализация приложения}

\subsubsection{Реализация шаблона проектирования}

В основе Django лежит шаблон проектирования MVC, который во фреймворке называется Model-View-Template, где Model --- модель, являющаяся фактически ORM-сущностью, View --- контроллер, Template --- представление. Бизнес-логику в Django принято выделять в отдельный компонент.

\subsubsection{Компоненты приложения}

Выделим три компонента: компонент доступа к данным, компонент графического интерфейса пользователя, компонент, связывающий данные и графический интерфейс пользователя.

Компонент доступа к данным представляет из себя классы \code{django.db.models.Models} (далее — модели), данные в которых соответствуют атрибутам таблиц в базе данных. Модели содержат методы для обработки данных на уровне строки, например метод \code{get\_to\_emails()}, возвращающий список адресатов сообщения. На листинге \ref{lst:model} представлена модель \code{Message}, соответствующая таблице Message в базе данных.

\lstinputlisting[language=python, caption={Модель \code{Message}}\label{lst:model}] {inc/lst/model.py}

Для работы с данными на уровне таблицы используются классы, называемые менеджерами. Эти классы наследуются от \code{django.db.models.Managers} и содержат методы для доступа к данным. На листинге \ref{lst:manager} приведен пример менеджера.

\lstinputlisting[language=python, caption={Менеджер \code{MessageManager}}\label{lst:manager}] {inc/lst/manager.py}

Классы-контроллеры в Django наследуются от класса \code{django.views.generic.DefaultView}. При этом существуют классы для выполнения типичных задач представления данных: для отображения списка объектов --- \code{ListView}, для отображения информации о конкретном объекте --- \code{DetailView} и др. На листинге \ref{lst:view} представлен класс \code{MessageDetailView}.

\lstinputlisting[language=python, caption={Контроллер \code{MessageDetailView}}\label{lst:view}] {inc/lst/view.py}

Компонент интерфейса представляет из себя набор HTML-страниц, использующих шаблонизатор Django, который позволяет использовать шаблоны для генерации конечных страниц. Шаблонизатор позволяет, в частности, переиспользовать код и ускоряет верстку веб-приложения. Данные передаются из контроллера в качестве параметров на страницу.

\subsection{Интерфейс}

Интерфейс приложения представляет из себя страницу, в шапке которой расположено название приложение, навигационное меню, позволяющее перейти к спискам входящих (Inbox) и исходящих сообщений (Outbox), информацию об авторизованном пользователе.

На странице Inbox расположена таблица, содержащая входящие письма. Каждое письмо представлено адресантом, темой письма и содержимым письма, которое при переполнении первой строки скрывается. Для каждого письма предусмотрена кнопка удаления. Структура страницы Outbox аналогична. При нажатии на письмо происходит переход на страницу Message, на которой отображена более подробная информация о письме: адресат и список адресантов, время получения письма, тема письма, полное содержимое письма. На каждой странице приложения справа от заголовка страницы расположена кнопка Compose, которая открывает модальное окно, позволяющее написать письмо.

На рисунках \ref{img:login}--\ref{img:compose} представлен интерфейс веб-приложения.

\imgwc{h}{\textwidth}{login}{Авторизация в веб-приложении}

\imgwc{h}{\textwidth}{inbox}{Вкладка входящих сообщений}

\imgwc{h}{\textwidth}{message}{Просмотр сообщения}

\imgwc{h}{\textwidth}{compose}{Создание сообщения}

\section{Тестирование и отладка}

Для программы использовался отладчик и статический анализатор кода, встроенные в среду разработки PyCharm.

Библиотеки, SMTP-сервер и веб-приложение были протестированы в полном объеме, все обнаруженные ошибки были исправлены.

\section{Выводы}

В результате разработки было реализовано программное обеспечение в полном соответствии с техническим заданием и предъявляемыми требованиями.